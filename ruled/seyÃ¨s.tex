% !TeX root = ./seyès.tex

% Overleaf detects the main file by \documentclass
\documentclass[UTF8,11pt,twoside]{ctexart}

% The invariant part of the preamble
% !TeX root = ./explain.tex

% Font and symbols
\usepackage[T1]{fontenc}
\usepackage{newpxtext,newpxmath}
\usepackage{textcomp}

% Tikz
\usepackage[svgnames]{xcolor}
\usepackage{tikz}
\usetikzlibrary{calc}
\usepackage{tikzpagenodes}

% Put hyperref Last
\usepackage{hyperref}
\hypersetup{hidelinks}

% Adjust printer and paper parameters in the following file
% !TeX root = ./explain.tex

% Define paper geometry and text areas; for more settings, see
% http://mirrors.ctan.org/macros/latex/contrib/geometry/geometry.pdf
\usepackage[
  % Paper size (a4paper, b5paper, letterpaper, etc.)
  letterpaper,
  % Orientation (portrait or landscape)
  portrait,
  % Text area margin
  top=0mm,
  bottom=0mm,
  left=0mm,
  right=0mm
]{geometry}

% Declare canvas margins within text area---if the canvas margins in one
% dimension are both set to 0 (meaning the canvas and the text area have the
% same length in that dimension), then body.tex will automatically center the
% picture within the canvas (and thus the text area) in that dimension unless
% the corresponding centering option is disabled in template
\pgfmathsetlengthmacro{\canvastop}{0mm}
\pgfmathsetlengthmacro{\canvasbottom}{5mm} % to ensure the last row is printed in its entirety
\pgfmathsetmacro{\canvasvcentering}{!true}
\pgfmathsetlengthmacro{\canvasleft}{0mm}
\pgfmathsetlengthmacro{\canvasright}{0mm}
\pgfmathsetmacro{\canvashcentering}{!true}

% Normally we want empty header and footer
\usepackage{fancyhdr}
\pagestyle{empty}

% Enable this option to show bounding boxes for debugging
\newif\ifshowbb
% \showbbtrue


% Compute \canvasheight and \canvaswidth
\pgfmathsetlengthmacro{\canvasheight}{\textheight-\canvastop-\canvasbottom}
\pgfmathsetlengthmacro{\canvaswidth}{\textwidth-\canvasleft-\canvasright}


% Declare template
\def\template{Seyès-ruled sheet}
\pgfmathsetmacro{\canvasvcentering}{!true}
\pgfmathsetmacro{\canvashcentering}{true}

% Declare offsets and separations of x- and y-stamps
\def\xoffset{16mm}
\def\xsep{8mm}
\def\yoffset{12mm}
\def\ysep{8mm}

% Declare guideline colors
\providecolor{scolorm}{cmyk:named}{black!20}
\providecolor{scolorn}{cmyk:named}{black!15}
\providecolor{scolorv}{cmyk:named}{black!20}

% Declare a tikz pic for stamping top to bottom
\tikzset{
  /s/h/major/.style=
    {
      scolorm,
      line width=0.2mm
    },
  /s/h/minor/.style=
    {
      scolorn,
      line width=0.1mm
    },
  pics/y-stamp/.style=
    {
      code=
        {
          \draw [/s/h/major]
          (0,    0) -- +(\picturewidth,0)
          (0,\ysep) -- +(\picturewidth,0);
          \draw [/s/h/minor]
          (0,1*\ysep/4) -- +(\picturewidth,0)
          (0,2*\ysep/4) -- +(\picturewidth,0)
          (0,3*\ysep/4) -- +(\picturewidth,0);
        }
    }
}

% Declare a tikz pic for stamping left to right
\tikzset{
  /s/v/.style=
    {
      scolorv,
      line width=0.2mm
    },
  pics/x-stamp/.style=
    {
      code=
        {
          \draw [/s/v]
          (0,0 |- canvas rect.north) -- (0,0 |- canvas rect.south);
        }
    }
}

% Compute stamping offsets, lengths, and separations
\pgfmathsetlengthmacro{\sxoff}{\xoffset}
\pgfmathsetlengthmacro{\sxlen}{0}
\pgfmathsetlengthmacro{\sxsep}{\xsep}
\pgfmathsetlengthmacro{\syoff}{\yoffset}
\pgfmathsetlengthmacro{\sylen}{\ysep}
\pgfmathsetlengthmacro{\sysep}{0}

% Compute the number of stamps in picture
\pgfmathsetmacro{\nx}{int((\canvaswidth-\sxoff)/\sxsep+1)}
\pgfmathsetmacro{\ny}{int((\canvasheight-\syoff+\sysep)/(\sylen+\sysep))}

% Compute picture height and width
\pgfmathsetlengthmacro{\pictureheight}{\syoff+\ny*(\sylen+\sysep)-\sysep}
\pgfmathsetlengthmacro{\picturewidth}{\canvaswidth}

% Set PDF properties
\providecolor{wmcolor}{cmyk:named}{black!0}
\extractcolorspec{scolorm}{\scolorm}
\extractcolorspec{scolorn}{\scolorn}
\extractcolorspec{scolorv}{\scolorv}
\def\pdftitle{%
  \ny-line \template{} (\ysep*\xsep)
}
\hypersetup{
  pdftitle=\pdftitle,
  pdfkeywords=
    {%
      \template;
      m=\scolorm; n=\scolorn;
      v=\scolorv
    },
  pdfsubject={https://github.com/maverickwoo/paperpad-templates},
  pdfauthor={Maverick Woo}
}

\providecolor{wmcolor}{cmyk:named}{black!20}
\def\pdftitle{%
}

% Go!
\begin{document}
% !TeX root = ./explain.tex
%
\ifshowbb\begin{tikzpicture}[remember picture, overlay, draw=black, very thick,
    dashed]

  % Draw rectangles around text areas
  \draw     (current page text area.south west)
  rectangle (current page text area.north east);
  \draw     (current page marginpar area.south west)
  rectangle (current page marginpar area.north east);
  \draw     (current page header area.south west)
  rectangle (current page header area.north east);
  \draw     (current page footer area.south west)
  rectangle (current page footer area.north east);

\end{tikzpicture}\fi
%
\begin{tikzpicture}[remember picture, overlay]

  % Add watermark into background
  \def\repourl{https://github.com/maverickwoo/paperpad-templates}
  \draw [inner sep=0pt, font=\ttfamily\tiny, color=wmcolor]
  node [anchor=north] at ($(current page.north) - (0mm,5mm)$) {
    \href{\repourl}{\number\year-\number\month-\number\day}}
  node [anchor=south] at ($(current page.south) + (0mm,5mm)$) {
    \href{\repourl}{\pdftitle}};

  % Create nodes "canvas rect" and "picture rect" using "current page text area"
  \begin{scope}[
      shift=(current page text area.north west),
      anchor=north west,
      line width=0, % node anchor points are on the outside of border lines
      fill opacity=\ifshowbb 0.5 \else 0 \fi]

    % Compute picture top-left (either center picture or anchor to canvas top-left)
    \pgfmathsetlengthmacro{\picturetop}{\canvasvcentering ?
      (\canvasheight-\pictureheight)/2 : \canvastop}
    \pgfmathsetlengthmacro{\pictureleft}{\canvashcentering ?
      (\canvaswidth-\picturewidth)/2 : \canvasleft}

    \node (canvas rect) [fill=cyan]
    at (\canvasleft, -\canvastop)
    [minimum width=\canvaswidth, minimum height=\canvasheight] {ct: \canvastop, pt: \picturetop};

    \node (picture rect) [fill=pink]
    at (\pictureleft, -\picturetop)
    [minimum width=\picturewidth, minimum height=\pictureheight] {ct: \canvastop, pt: \picturetop};

  \end{scope}

  % Draw target picture
  % (depend on picture rect, ny, nx, sylen, sysep, y-stamp, ...)
  \begin{scope}[shift=(picture rect.south west)]

    % Print y-stamps from top to bottom along picture west
    \foreach \iy in {\synum,...,1}
      {
        \pgfmathsetlengthmacro{\xcoord}{0}
        \pgfmathsetlengthmacro{\ycoord}{(\iy   -  1)*(\sylen+\sysep)}
        \pgfmathsetlengthmacro{\ydepth}{(\synum-\iy)*(\sylen+\sysep)}
        \draw (\xcoord,\ycoord) pic {y-stamp};
      }

    % Start clipping on picture east and west for x-stamps
    \clip     (\picturewidth,0 |- current page.north)
    rectangle (            0,0 |- current page.south);

    % Print x-stamps from left to right along picture north
    \foreach \ix in {1,...,\sxnum}
      {
        \pgfmathsetlengthmacro{\xcoord}{\sxoff+(\ix-1)*\sxsep}
        \pgfmathsetlengthmacro{\ycoord}{\pictureheight}
        \draw (\xcoord,\ycoord) pic {x-stamp};
      }
  \end{scope}

\end{tikzpicture}\ignorespaces

\end{document}
