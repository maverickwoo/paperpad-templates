% !TeX root = ./explain.tex
%
\ifshowbb\begin{tikzpicture}[remember picture, overlay, green, fill opacity=0.1]

  % Shade text areas
  \fill     (current page text area.south west)
  rectangle (current page text area.north east);
  \fill     (current page marginpar area.south west)
  rectangle (current page marginpar area.north east);
  \fill     (current page header area.south west)
  rectangle (current page header area.north east);
  \fill     (current page footer area.south west)
  rectangle (current page footer area.north east);

\end{tikzpicture}\fi
%
\begin{tikzpicture}[remember picture, overlay]

  % Add watermark into background
  \def\repourl{https://github.com/maverickwoo/paperpad-templates}
  \draw [inner sep=0pt, font=\ttfamily\tiny, color=wmcolor]
  node [anchor=north] at ($(current page.north) - (0mm,5mm)$) {
    \href{\repourl}{\number\year-\number\month-\number\day}}
  node [anchor=south] at ($(current page.south) + (0mm,5mm)$) {
    \href{\repourl}{\pdftitle}};

  % Create nodes "canvas rect" and "picture rect" using "current page text area"
  \begin{scope}[
      shift=(current page text area.north west),
      anchor=north west,
      line width=0, % node anchor points are on the outside of border lines
      fill opacity=\ifshowbb 0.1 \else 0 \fi]

    % Compute picture top-left (either center picture or anchor to canvas top-left)
    \pgfmathparse{\canvasvcentering && \canvastop==0 && \canvasbottom==0}
    \ifnum\pgfmathresult>0
      \pgfmathsetlengthmacro{\picturetop}{(\textheight-\pictureheight)/2}
    \else
      \pgfmathsetlengthmacro{\picturetop}{\canvastop}
    \fi
    \pgfmathparse{\canvashcentering && \canvasleft==0 && \canvasright==0}
    \ifnum\pgfmathresult>0
      \pgfmathsetlengthmacro{\pictureleft}{(\textwidth-\picturewidth)/2}
    \else
      \pgfmathsetlengthmacro{\pictureleft}{\canvasleft}
    \fi

    \node (canvas rect) [fill=cyan]
    at (\canvasleft, -\canvastop)
    [minimum width=\canvaswidth, minimum height=\canvasheight] {};

    \node (picture rect) [fill=pink]
    at (\pictureleft, -\picturetop)
    [minimum width=\picturewidth, minimum height=\pictureheight] {};

  \end{scope}

  % Draw target picture
  \begin{scope}[shift=(picture rect.south west)]

    % Print y-stamps from top to bottom along picture west
    \foreach \iy in {\ny,...,1}
      {
        \pgfmathsetlengthmacro{\xcoord}{0}
        \pgfmathsetlengthmacro{\ycoord}{(\iy-  1)*(\sylen+\sysep)}
        \pgfmathsetlengthmacro{\ydepth}{(\ny-\iy)*(\sylen+\sysep)}
        \draw (\xcoord,\ycoord) pic {y-stamp};
      }

    % Start clipping on picture east and west for x-stamps
    \clip     (\picturewidth,0 |- current page.north)
    rectangle (            0,0 |- current page.south);

    % Print x-stamps from left to right along picture north
    \foreach \ix in {1,...,\nx}
      {
        \pgfmathsetlengthmacro{\xcoord}{\sxoff+(\ix-1)*\sxsep}
        \pgfmathsetlengthmacro{\ycoord}{\pictureheight}
        \draw (\xcoord,\ycoord) pic {x-stamp};
      }
  \end{scope}

\end{tikzpicture}\ignorespaces
