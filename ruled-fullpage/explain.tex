% !TeX root = ./explain.tex

% Overleaf detects the main file by \documentclass
\documentclass[UTF8,11pt,twoside]{ctexart}

% The invariant part of the preamble
% !TeX root = ./slant.tex

\documentclass[UTF8,11pt]{ctexart}

% Adjust printer and paper parameters in the following file
% !TeX root = ./explain.tex

% Define paper geometry and text areas; for more settings, see
% http://mirrors.ctan.org/macros/latex/contrib/geometry/geometry.pdf
\usepackage[
  % Paper size (a4paper, b5paper, letterpaper, etc.)
  letterpaper,
  % Orientation (portrait or landscape)
  portrait,
  % Text area margin
  top=5mm,
  bottom=5mm,
  left=4mm,
  right=4mm
]{geometry}

% Declare canvas margins within text area---if the canvas margins in one
% dimension are both set to 0 (meaning the canvas and the text area have the
% same length in that dimension), then body.tex will automatically center the
% picture within the canvas (and thus the text area) in that dimension unless
% the corresponding centering option is disabled in template
\pgfmathsetlengthmacro{\canvastop}{0mm}
\pgfmathsetlengthmacro{\canvasbottom}{0mm}
\pgfmathsetlengthmacro{\canvasleft}{0mm}
\pgfmathsetlengthmacro{\canvasright}{0mm}

% Normally we want empty header and footer
\usepackage{fancyhdr}
\pagestyle{empty}

% Enable this option to show bounding boxes for debugging
\newif\ifshowbb
% \showbbtrue


\showbbtrue

% Declare template
\def\template{Explanation for Ruled-Paper Template}

% Declare a tikz pic for stamping top to bottom
\tikzset{
  pics/y-stamp/.style=
    {
      code=
        {
          \draw (0,0) circle (1mm)
          node at (0,0) [pin=above right:origin of first y-stamp] {};
        }
    }
}

% Declare a tikz pic for stamping left to right
\tikzset{
  pics/x-stamp/.style=
    {
      code=
        {
          \draw (0,0) circle (1mm)
          node at (0,0) [pin=below right:origin of first x-stamp] {};
        }
    }
}

% Compute stamping offsets, lengths, and separations
\pgfmathsetlengthmacro{\sxoff}{1in}
\pgfmathsetlengthmacro{\sxlen}{0}
\pgfmathsetlengthmacro{\sxsep}{0}
\pgfmathsetlengthmacro{\syoff}{1in}
\pgfmathsetlengthmacro{\sylen}{0}
\pgfmathsetlengthmacro{\sysep}{0}

% Compute the number of stamps in picture
\pgfmathsetmacro{\sxnum}{1}
\pgfmathsetmacro{\synum}{1}

% Compute picture height and width
\pgfmathsetlengthmacro{\pictureheight}{\canvasheight}
\pgfmathsetlengthmacro{\picturewidth}{\canvaswidth}

% Set PDF properties
\providecolor{wmcolor}{cmyk:named}{black!0}
\def\pdftitle{\template}
\hypersetup{
  pdftitle=\pdftitle,
  pdfsubject={https://github.com/maverickwoo/paperpad-templates},
  pdfauthor={Maverick Woo}
}

% Go!
\begin{document}

\section{Introduction}

To understand this template, you need to understand two groups of concepts: (i)
geometry and (ii) stamping.

% !TeX root = ./explain.tex
% This file is not intended to contain any user-adjustable parameters.

\begin{document}

\begin{tikzpicture}[remember picture,overlay]

  % Add watermark
  \node [font=\ttfamily\tiny,color=black!20]
  at ($(current page text area.north) - (0mm,0.5ex)$) {
    \href{https://github.com/maverickwoo/paperpad-templates}{\number\year-\number\month-\number\day}
  };
  \node [font=\ttfamily\tiny,color=black!20]
  at ($(current page text area.south) + (0mm,0.5ex)$) {
    \href{https://github.com/maverickwoo/paperpad-templates}{\pdftitle}
  };

  % Compute origin coordinate
  \pgfmathsetlengthmacro{\totalwidth}{\textwidth}
  \pgfmathsetlengthmacro{\marginx}{(\paperwidth-\totalwidth)/2}
  \pgfmathsetlengthmacro{\totalheight}{\ny*\sylen+(\ny-1)*\sysep+\syoff}
  \pgfmathsetlengthmacro{\marginy}{(\paperheight-\totalheight)/2}
  \coordinate (origin) at ($(current page.south west) + (\marginx,\marginy)$);

  % Print y-stamps from top to bottom
  \foreach \iy in {\ny,...,1}
    {
      % depth of current ascender for dash phase computations
      \pgfmathsetlengthmacro{\depthy}{(\ny-\iy)*(\sylen+\sysep)}
      \pgfmathsetlengthmacro{\deltay}{(\iy-  1)*(\sylen+\sysep)}
      \draw ($(origin)+(0,\deltay)$) pic {ystamp};
    }

  % Clip slanted guides at horizontal margins
  \clip ($(origin)-(0,\marginy)$)
  rectangle ($(origin)+(0,\marginy)+(\totalwidth,\totalheight)$);

  % Print x-stamps from left to right
  \foreach \ix in {1,...,\nx}
    {
      \pgfmathsetlengthmacro{\deltax}{(\ix-1)*\sxsep}
      \draw ($(origin)+(\deltax,\totalheight)$) pic {xstamp};
    }

\end{tikzpicture}

\end{document}


\end{document}
