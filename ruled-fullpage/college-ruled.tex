% !TeX root = ./college-ruled.tex

% Overleaf detects the main file by \documentclass
\documentclass[UTF8,11pt,twoside]{ctexart}

% The invariant part of the preamble
% !TeX root = ./slant.tex

\documentclass[UTF8,11pt]{ctexart}

% Adjust printer and paper parameters in the following file
% !TeX root = ./explain.tex

% Define paper geometry and text areas; for more settings, see
% http://mirrors.ctan.org/macros/latex/contrib/geometry/geometry.pdf
\usepackage[
  % Paper size (a4paper, b5paper, letterpaper, etc.)
  letterpaper,
  % Orientation (portrait or landscape)
  portrait,
  % Text area margin
  top=5mm,
  bottom=5mm,
  left=4mm,
  right=4mm
]{geometry}

% Declare canvas margins within text area---if the canvas margins in one
% dimension are both set to 0 (meaning the canvas and the text area have the
% same length in that dimension), then body.tex will automatically center the
% picture within the canvas (and thus the text area) in that dimension unless
% the corresponding centering option is disabled in template
\pgfmathsetlengthmacro{\canvastop}{0mm}
\pgfmathsetlengthmacro{\canvasbottom}{0mm}
\pgfmathsetlengthmacro{\canvasleft}{0mm}
\pgfmathsetlengthmacro{\canvasright}{0mm}

% Normally we want empty header and footer
\usepackage{fancyhdr}
\pagestyle{empty}

% Enable this option to show bounding boxes for debugging
\newif\ifshowbb
% \showbbtrue



% Declare template
\def\template{College-ruled sheet}

% Declare offsets and separations of x- and y-stamps
\def\xoffset{1in+2in/16} % 5/16 is Atoma
\def\xsep{\textwidth}
\def\yoffset{1in-1in/16} % 1/16 is printer paper feed inaccuracy
\def\ysep{9in/32}

% Declare guideline colors
\providecolor{scolorh}{cmyk:named}{cyan!66!yellow!40}
\providecolor{scolorv}{cmyk:named}{magenta!66!yellow!60}

% Declare a tikz pic for stamping top to bottom
\tikzset{
  /s/h/.style=
    {
      scolorh,
      line width=0.2mm
    },
  pics/y-stamp/.style=
    {
      code=
        {
          \draw [/s/h]
          (0,    0) -- +(\picturewidth,0)
          (0,\ysep) -- +(\picturewidth,0);

          \draw (0,0) circle (5mm);
          \node at (5cm,0) {ch: \canvasheight, ph: \pictureheight};
        }
    }
}

% Declare a tikz pic for stamping left to right
\tikzset{
  /s/v/.style=
    {
      scolorv,
      line width=0.2mm
    },
  pics/x-stamp/.style=
    {
      code=
        {
          \draw [/s/v]
          (0,0 |- canvas rect.north)
          -- (0,0 |- canvas rect.south);

          \draw (0,0) circle (5mm);
        }
    }
}

% Compute stamping offsets, lengths, and separations
\pgfmathsetlengthmacro{\sxoff}{\xoffset}
\pgfmathsetlengthmacro{\sxlen}{0}
\pgfmathsetlengthmacro{\sxsep}{\xsep}
\pgfmathsetlengthmacro{\syoff}{\yoffset}
\pgfmathsetlengthmacro{\sylen}{\ysep}
\pgfmathsetlengthmacro{\sysep}{0}

% Compute the number of stamps in picture
% todo: the reason why this does not generalize to slanted is because we need to extend the width
\pgfmathsetmacro{\sxnum}{int((\canvaswidth-\sxoff+\sxsep)/(\sxlen+\sxsep))}
\pgfmathsetmacro{\synum}{int((\canvasheight-\syoff+\sysep)/(\sylen+\sysep))}

% Compute picture height and width
\pgfmathsetlengthmacro{\pictureheight}{\syoff+\synum*(\sylen+\sysep)-\sysep}
\pgfmathsetlengthmacro{\picturewidth}{\canvaswidth}

% Set PDF properties
\providecolor{wmcolor}{cmyk:named}{black!0}
\extractcolorspec{scolorh}{\scolorh}
\extractcolorspec{scolorv}{\scolorv}
\def\pdftitle{%
  \synum-line \template{} (\ysep)
}
\hypersetup{
  pdftitle=\pdftitle,
  pdfkeywords=
    {%
      \template;
      h=\scolorh; v=\scolorv
    },
  pdfsubject={https://github.com/maverickwoo/paperpad-templates},
  pdfauthor={Maverick Woo}
}

\providecolor{wmcolor}{cmyk:named}{black!20}
\def\pdftitle{%
}

% Go!
\begin{document}
% !TeX root = ./explain.tex
% This file is not intended to contain any user-adjustable parameters.

\begin{document}

\begin{tikzpicture}[remember picture,overlay]

  % Add watermark
  \node [font=\ttfamily\tiny,color=black!20]
  at ($(current page text area.north) - (0mm,0.5ex)$) {
    \href{https://github.com/maverickwoo/paperpad-templates}{\number\year-\number\month-\number\day}
  };
  \node [font=\ttfamily\tiny,color=black!20]
  at ($(current page text area.south) + (0mm,0.5ex)$) {
    \href{https://github.com/maverickwoo/paperpad-templates}{\pdftitle}
  };

  % Compute origin coordinate
  \pgfmathsetlengthmacro{\totalwidth}{\textwidth}
  \pgfmathsetlengthmacro{\marginx}{(\paperwidth-\totalwidth)/2}
  \pgfmathsetlengthmacro{\totalheight}{\ny*\sylen+(\ny-1)*\sysep+\syoff}
  \pgfmathsetlengthmacro{\marginy}{(\paperheight-\totalheight)/2}
  \coordinate (origin) at ($(current page.south west) + (\marginx,\marginy)$);

  % Print y-stamps from top to bottom
  \foreach \iy in {\ny,...,1}
    {
      % depth of current ascender for dash phase computations
      \pgfmathsetlengthmacro{\depthy}{(\ny-\iy)*(\sylen+\sysep)}
      \pgfmathsetlengthmacro{\deltay}{(\iy-  1)*(\sylen+\sysep)}
      \draw ($(origin)+(0,\deltay)$) pic {ystamp};
    }

  % Clip slanted guides at horizontal margins
  \clip ($(origin)-(0,\marginy)$)
  rectangle ($(origin)+(0,\marginy)+(\totalwidth,\totalheight)$);

  % Print x-stamps from left to right
  \foreach \ix in {1,...,\nx}
    {
      \pgfmathsetlengthmacro{\deltax}{(\ix-1)*\sxsep}
      \draw ($(origin)+(\deltax,\totalheight)$) pic {xstamp};
    }

\end{tikzpicture}

\end{document}

\end{document}
