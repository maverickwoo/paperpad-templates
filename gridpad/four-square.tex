% !TeX root = ./four-square.tex
\documentclass[UTF8]{ctexart}

% Adjust paper and printer parameters in the following file
% Choose your paper type and use the margins to specify the unprintable area of
% your printer
\usepackage[
  letterpaper, % a common alternative is a4paper
  %landscape, % default is portrait; uncomment this line to turn into landscape
  top=5mm,
  bottom=5mm,
  left=4mm,
  right=4mm
]{geometry}

%%%%%%%%%%%%%%%%%%%%%%%%%%%%%%%%%%%%%%%%%%%%%%%%%%%%%%%%%%%%%%%%%%%%%%%%%%%%%%%%

% Everything below this line is not intended to be user-configurable
\usepackage[svgnames]{xcolor}
\usepackage[T1]{fontenc}
\usepackage{tikz}
\usetikzlibrary{calc}
\usepackage{tikzpagenodes}
\usepackage{hyperref}
\hypersetup{hidelinks}


% Name the grid
\def\gridname{four-square grid (田字格)}

% Declare stamp dimensions and separations
\def\sxlen{11mm}\def\sxsep{0mm}
\def\sylen{11mm}\def\sysep{0mm}

% Declare a tikz pic for stamping onto the grid
\providecolor{scolora}{cmyk:named}{magenta!66!yellow!50}
\providecolor{scolorb}{cmyk:named}{cyan!66!yellow!60}
\pgfmathsetlengthmacro{\sxdash}{\sxlen/3/2/2}
\pgfmathsetlengthmacro{\sydash}{\sylen/3/2/2}
\tikzset{
  /s/major/x/.style = {
      scolora,
      line cap=rect,
      line width=0.3mm
    },
  /s/major/y/.style = {
      scolora,
      line cap=rect,
      line width=0.3mm
    },
  /s/minor/x/.style = {
      scolorb,
      line cap=butt,
      line width=0.15mm,
      dash pattern=on \sxdash off \sxdash,
      dash phase=0.5*\sxdash % on-center: 0.5; off-center: 1.5
    },
  /s/minor/y/.style = {
      scolorb,
      line cap=butt,
      line width=0.15mm,
      dash pattern=on \sydash off \sydash,
      dash phase=0.5*\sydash % on-center: 0.5; off-center: 1.5
    },
  pics/stamp/.style={
      code={
          \draw [/s/minor/x] (0,\sylen/2) -- +(\sxlen,0);
          \draw [/s/minor/y] (\sxlen/2,0) -- +(0,\sylen);

          \draw [/s/major/x] (0,     0) -- +(\sxlen,0);
          \draw [/s/major/x] (0,\sylen) -- +(\sxlen,0);
          \draw [/s/major/y] (     0,0) -- +(0,\sylen);
          \draw [/s/major/y] (\sxlen,0) -- +(0,\sylen);
        }
    }
}

% Compute the number of stamps that can fit in printable area
\pgfmathsetmacro{\nx}{int((\textwidth +\sxsep)/(\sxlen+\sxsep))}
\pgfmathsetmacro{\ny}{int((\textheight+\sysep)/(\sylen+\sysep))}

% Set PDF properties
\extractcolorspec{scolora}{\scolora}
\extractcolorspec{scolorb}{\scolorb}
\def\pdftitle{\ny*\nx{} \gridname{} of \sylen-by-\sxlen{} squares}
\hypersetup{
  pdftitle=\pdftitle,
  pdfkeywords={%
      \ny*\nx{} \gridname,
      h: \sylen{} + \sysep,
      w: \sxlen{} + \sxsep,
      c: \{\scolora; \scolorb\}
    },
  pdfsubject={https://github.com/maverickwoo/paperpad-templates},
  pdfauthor={Maverick Woo}
}

% !TeX root = ./explain.tex
%
\ifshowbb\begin{tikzpicture}[remember picture, overlay, draw=black, very thick,
    dashed]

  % Draw rectangles around text areas
  \draw     (current page text area.south west)
  rectangle (current page text area.north east);
  \draw     (current page marginpar area.south west)
  rectangle (current page marginpar area.north east);
  \draw     (current page header area.south west)
  rectangle (current page header area.north east);
  \draw     (current page footer area.south west)
  rectangle (current page footer area.north east);

\end{tikzpicture}\fi
%
\begin{tikzpicture}[remember picture, overlay]

  % Add watermark into background
  \def\repourl{https://github.com/maverickwoo/paperpad-templates}
  \draw [inner sep=0pt, font=\ttfamily\tiny, color=wmcolor]
  node [anchor=north] at ($(current page.north) - (0mm,5mm)$) {
    \href{\repourl}{\number\year-\number\month-\number\day}}
  node [anchor=south] at ($(current page.south) + (0mm,5mm)$) {
    \href{\repourl}{\pdftitle}};

  % Create nodes "canvas rect" and "picture rect" using "current page text area"
  \begin{scope}[
      shift=(current page text area.north west),
      anchor=north west,
      line width=0, % node anchor points are on the outside of border lines
      fill opacity=\ifshowbb 0.5 \else 0 \fi]

    % Compute picture top-left (either center picture or anchor to canvas top-left)
    \pgfmathsetlengthmacro{\picturetop}{\canvasvcentering ?
      (\canvasheight-\pictureheight)/2 : \canvastop}
    \pgfmathsetlengthmacro{\pictureleft}{\canvashcentering ?
      (\canvaswidth-\picturewidth)/2 : \canvasleft}

    \node (canvas rect) [fill=cyan]
    at (\canvasleft, -\canvastop)
    [minimum width=\canvaswidth, minimum height=\canvasheight] {ct: \canvastop, pt: \picturetop};

    \node (picture rect) [fill=pink]
    at (\pictureleft, -\picturetop)
    [minimum width=\picturewidth, minimum height=\pictureheight] {ct: \canvastop, pt: \picturetop};

  \end{scope}

  % Draw target picture
  % (depend on picture rect, ny, nx, sylen, sysep, y-stamp, ...)
  \begin{scope}[shift=(picture rect.south west)]

    % Print y-stamps from top to bottom along picture west
    \foreach \iy in {\synum,...,1}
      {
        \pgfmathsetlengthmacro{\xcoord}{0}
        \pgfmathsetlengthmacro{\ycoord}{(\iy   -  1)*(\sylen+\sysep)}
        \pgfmathsetlengthmacro{\ydepth}{(\synum-\iy)*(\sylen+\sysep)}
        \draw (\xcoord,\ycoord) pic {y-stamp};
      }

    % Start clipping on picture east and west for x-stamps
    \clip     (\picturewidth,0 |- current page.north)
    rectangle (            0,0 |- current page.south);

    % Print x-stamps from left to right along picture north
    \foreach \ix in {1,...,\sxnum}
      {
        \pgfmathsetlengthmacro{\xcoord}{\sxoff+(\ix-1)*\sxsep}
        \pgfmathsetlengthmacro{\ycoord}{\pictureheight}
        \draw (\xcoord,\ycoord) pic {x-stamp};
      }
  \end{scope}

\end{tikzpicture}\ignorespaces

