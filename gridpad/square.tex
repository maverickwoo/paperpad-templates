% !TeX root = ./square.tex
\documentclass[UTF8]{ctexart}

% Adjust paper and printer parameters in the following file
% Choose paper type and specify unprintable margins of your printer
\usepackage[
  letterpaper,
  %landscape, % default=portrait; uncomment to rotate (be sure to fix margin!)
  top=5mm,
  bottom=5mm,
  left=4mm,
  right=4mm
]{geometry}

%%%%%%%%%%%%%%%%%%%%%%%%%%%%%%%%%%%%%%%%%%%%%%%%%%%%%%%%%%%%%%%%%%%%%%%%%%%%%%%%

% Everything below this line is not intended to be user-configurable
\usepackage{libertine}
\usepackage{libertinust1math}
\usepackage[T1]{fontenc}
\usepackage{textcomp}
\usepackage[svgnames]{xcolor}
\usetikzlibrary{calc}
\usepackage{tikzpagenodes}
\usepackage{hyperref}
\hypersetup{hidelinks}


% Name the grid
\def\gridname{square grid (正方格)}

% Declare stamp dimensions and separations
\def\stampheight{13mm}\def\stampvsep{0mm}
\def\stampwidth {13mm}\def\stamphsep{0mm}

% Declare a tikz pic for stamping onto the grid
\providecolor{stampcolora}{cmyk:named}{cyan!66!yellow}
\tikzset{
  stamp majorline/.style = {
      stampcolora,
      line cap=rect,
      line width=0.3mm
    },
  pics/stamp/.style={
      code={
          % Horizontal
          \draw [stamp majorline] (0,           0) -- (\stampwidth,           0);
          \draw [stamp majorline] (0,\stampheight) -- (\stampwidth,\stampheight);
          % Vertical
          \draw [stamp majorline] (          0,0) -- (          0,\stampheight);
          \draw [stamp majorline] (\stampwidth,0) -- (\stampwidth,\stampheight);
        }
    }
}

% Compute the number of stamps that can fit in printable area
\pgfmathsetmacro{\nx}{int((\textwidth +\stamphsep)/(\stampwidth +\stamphsep))}
\pgfmathsetmacro{\ny}{int((\textheight+\stampvsep)/(\stampheight+\stampvsep))}

% Set PDF properties
\extractcolorspec{stampcolora}{\stampcolora}
\def\pdftitle{\ny*\nx{} \gridname{} of \stampheight-by-\stampwidth{} squares}
\hypersetup{
  pdftitle=\pdftitle,
  pdfkeywords={%
      \ny*\nx{} \gridname,
      h,s: \stampheight{} | \stampvsep,
      w,s: \stampwidth{} | \stamphsep,
      c: \{\stampcolora\}
    },
  pdfsubject={https://github.com/maverickwoo/paperpad-templates},
  pdfauthor={Maverick Woo}
}

% !TeX root = ./explain.tex
% This file is not intended to contain any user-adjustable parameters.

\begin{document}

\begin{tikzpicture}[remember picture,overlay]

  % Add watermark
  \node [font=\ttfamily\tiny,color=black!20]
  at ($(current page text area.north) - (0mm,0.5ex)$) {
    \href{https://github.com/maverickwoo/paperpad-templates}{\number\year-\number\month-\number\day}
  };
  \node [font=\ttfamily\tiny,color=black!20]
  at ($(current page text area.south) + (0mm,0.5ex)$) {
    \href{https://github.com/maverickwoo/paperpad-templates}{\pdftitle}
  };

  % Compute origin coordinate
  \pgfmathsetlengthmacro{\totalwidth}{\textwidth}
  \pgfmathsetlengthmacro{\marginx}{(\paperwidth-\totalwidth)/2}
  \pgfmathsetlengthmacro{\totalheight}{\ny*\sylen+(\ny-1)*\sysep+\syoff}
  \pgfmathsetlengthmacro{\marginy}{(\paperheight-\totalheight)/2}
  \coordinate (origin) at ($(current page.south west) + (\marginx,\marginy)$);

  % Print y-stamps from top to bottom
  \foreach \iy in {\ny,...,1}
    {
      % depth of current ascender for dash phase computations
      \pgfmathsetlengthmacro{\depthy}{(\ny-\iy)*(\sylen+\sysep)}
      \pgfmathsetlengthmacro{\deltay}{(\iy-  1)*(\sylen+\sysep)}
      \draw ($(origin)+(0,\deltay)$) pic {ystamp};
    }

  % Clip slanted guides at horizontal margins
  \clip ($(origin)-(0,\marginy)$)
  rectangle ($(origin)+(0,\marginy)+(\totalwidth,\totalheight)$);

  % Print x-stamps from left to right
  \foreach \ix in {1,...,\nx}
    {
      \pgfmathsetlengthmacro{\deltax}{(\ix-1)*\sxsep}
      \draw ($(origin)+(\deltax,\totalheight)$) pic {xstamp};
    }

\end{tikzpicture}

\end{document}

