% !TeX root = ./dotpad.tex
\documentclass[UTF8]{ctexart}

% Adjust paper and printer parameters in the following file
% Choose your paper type and use the margins to specify the unprintable area of
% your printer
\usepackage[
  letterpaper, % a common alternative is a4paper
  %landscape, % default is portrait; uncomment this line to turn into landscape
  top=5mm,
  bottom=5mm,
  left=4mm,
  right=4mm
]{geometry}

%%%%%%%%%%%%%%%%%%%%%%%%%%%%%%%%%%%%%%%%%%%%%%%%%%%%%%%%%%%%%%%%%%%%%%%%%%%%%%%%

% Everything below this line is not intended to be user-configurable
\usepackage[svgnames]{xcolor}
\usepackage[T1]{fontenc}
\usepackage{tikz}
\usetikzlibrary{calc}
\usepackage{tikzpagenodes}
\usepackage{hyperref}
\hypersetup{hidelinks}


% Name the grid
\def\gridname{dot pad}

% Declare stamp dimensions and separations
\def\stampheight{0mm}\def\stampvsep{5mm}
\def\stampwidth {0mm}\def\stamphsep{5mm}

% Declare a tikz pic for stamping onto the grid
\providecolor{stampcolor}{cmyk:named}{black!40}
\def\dotradius{0.2mm}
\tikzset{
  pics/stamp/.style={
      code={
          \fill circle (\dotradius)
          [
              color=stampcolor,
              inner sep=0mm,
              outer sep=0mm
            ];
        }
    }
}

% Compute the number of stamps that can fit in printable area
\pgfmathsetmacro{\nx}{int((\textwidth +\stamphsep)/(\stampwidth +\stamphsep))}
\pgfmathsetmacro{\ny}{int((\textheight+\stampvsep)/(\stampheight+\stampvsep))}

% Set PDF properties
\extractcolorspec{stampcolor}{\stampcolor}
\def\pdftitle{\ny*\nx{} \gridname{} with \stampvsep-by-\stamphsep{} spaces}
\hypersetup{
  pdftitle=\pdftitle,
  pdfkeywords={%
      \ny*\nx{} \gridname,
      h,s: \stampheight{} | \stampvsep,
      w,s: \stampwidth{} | \stamphsep,
      r: \dotradius,
      c: \{\stampcolor\}
    },
  pdfsubject={https://github.com/maverickwoo/paperpad-templates},
  pdfauthor={Maverick Woo}
}

% !TeX root = ./explain.tex
%
\ifshowbb\begin{tikzpicture}[remember picture, overlay, draw=black, very thick,
    dashed]

  % Draw rectangles around text areas
  \draw     (current page text area.south west)
  rectangle (current page text area.north east);
  \draw     (current page marginpar area.south west)
  rectangle (current page marginpar area.north east);
  \draw     (current page header area.south west)
  rectangle (current page header area.north east);
  \draw     (current page footer area.south west)
  rectangle (current page footer area.north east);

\end{tikzpicture}\fi
%
\begin{tikzpicture}[remember picture, overlay]

  % Add watermark into background
  \def\repourl{https://github.com/maverickwoo/paperpad-templates}
  \draw [inner sep=0pt, font=\ttfamily\tiny, color=wmcolor]
  node [anchor=north] at ($(current page.north) - (0mm,5mm)$) {
    \href{\repourl}{\number\year-\number\month-\number\day}}
  node [anchor=south] at ($(current page.south) + (0mm,5mm)$) {
    \href{\repourl}{\pdftitle}};

  % Create nodes "canvas rect" and "picture rect" using "current page text area"
  \begin{scope}[
      shift=(current page text area.north west),
      anchor=north west,
      line width=0, % node anchor points are on the outside of border lines
      fill opacity=\ifshowbb 0.5 \else 0 \fi]

    % Compute picture top-left (either center picture or anchor to canvas top-left)
    \pgfmathsetlengthmacro{\picturetop}{\canvasvcentering ?
      (\canvasheight-\pictureheight)/2 : \canvastop}
    \pgfmathsetlengthmacro{\pictureleft}{\canvashcentering ?
      (\canvaswidth-\picturewidth)/2 : \canvasleft}

    \node (canvas rect) [fill=cyan]
    at (\canvasleft, -\canvastop)
    [minimum width=\canvaswidth, minimum height=\canvasheight] {ct: \canvastop, pt: \picturetop};

    \node (picture rect) [fill=pink]
    at (\pictureleft, -\picturetop)
    [minimum width=\picturewidth, minimum height=\pictureheight] {ct: \canvastop, pt: \picturetop};

  \end{scope}

  % Draw target picture
  % (depend on picture rect, ny, nx, sylen, sysep, y-stamp, ...)
  \begin{scope}[shift=(picture rect.south west)]

    % Print y-stamps from top to bottom along picture west
    \foreach \iy in {\synum,...,1}
      {
        \pgfmathsetlengthmacro{\xcoord}{0}
        \pgfmathsetlengthmacro{\ycoord}{(\iy   -  1)*(\sylen+\sysep)}
        \pgfmathsetlengthmacro{\ydepth}{(\synum-\iy)*(\sylen+\sysep)}
        \draw (\xcoord,\ycoord) pic {y-stamp};
      }

    % Start clipping on picture east and west for x-stamps
    \clip     (\picturewidth,0 |- current page.north)
    rectangle (            0,0 |- current page.south);

    % Print x-stamps from left to right along picture north
    \foreach \ix in {1,...,\sxnum}
      {
        \pgfmathsetlengthmacro{\xcoord}{\sxoff+(\ix-1)*\sxsep}
        \pgfmathsetlengthmacro{\ycoord}{\pictureheight}
        \draw (\xcoord,\ycoord) pic {x-stamp};
      }
  \end{scope}

\end{tikzpicture}\ignorespaces

