% !TeX root = ./slanted.tex

\documentclass[UTF8,11pt]{ctexart}

% Adjust paper and printer parameters in the following file
% Choose your paper type and use the margins to specify the unprintable area of
% your printer
\usepackage[
  letterpaper, % a common alternative is a4paper
  %landscape, % default is portrait; uncomment this line to turn into landscape
  top=5mm,
  bottom=5mm,
  left=4mm,
  right=4mm
]{geometry}

%%%%%%%%%%%%%%%%%%%%%%%%%%%%%%%%%%%%%%%%%%%%%%%%%%%%%%%%%%%%%%%%%%%%%%%%%%%%%%%%

% Everything below this line is not intended to be user-configurable
\usepackage[svgnames]{xcolor}
\usepackage[T1]{fontenc}
\usepackage{tikz}
\usetikzlibrary{calc}
\usepackage{tikzpagenodes}
\usepackage{hyperref}
\hypersetup{hidelinks}


% Declare template name
\def\template{ruled lines with slant guidelines}

% Declare nib width as a basic unit
\def\nibwidth{0.8mm}

% Declare horizontal guidelines using letter proportions and nib width
\def\ascenderNW{9}
\def\capNW{8}
\def\xheightNW{4}
\def\descenderNW{5}
\def\yoffsetNW{0}
\def\ysepNW{0}

% Declare slant guidelines using nib width
\def\slantdegree{65}
\def\xoffsetNW{5.5}
\def\xsepNW{9}

% Declare line colors
\providecolor{scolora}{cmyk:named}{cyan!66!yellow!40}
\providecolor{scolorc}{cmyk:named}{black!40}
\providecolor{scolorx}{cmyk:named}{black!15}
\providecolor{scolorb}{cmyk:named}{black!15}
\providecolor{scolord}{cmyk:named}{cyan!66!yellow!40}
\providecolor{scolors}{cmyk:named}{black!40}

% Declare a tikz pic for stamping top to bottom; uses \depthy
\tikzset{
  /s/ascender/.style=
    {
      scolora,
      line cap=butt,
      line width=0.2mm,
      dash pattern=on 2*\nibwidth off \nibwidth,
      dash phase=\depthy/tan(\slantdegree)
    },
  /s/cap/.style=
    {
      scolorc,
      line cap=round,
      line width=0.2mm,
      dash pattern=on 0*\nibwidth off \nibwidth,
      dash phase=(\depthy+(\ascenderNW-\capNW)*\nibwidth)/tan(\slantdegree)+\nibwidth/2
    },
  /s/xheight/.style=
    {
      scolorx,
      line cap=butt,
      line width=0.1mm
    },
  /s/baseline/.style=
    {
      scolorb,
      line cap=butt,
      line width=0.1mm
    },
  /s/basecircle/.style=
    {
      scolorb,
      fill=white,
      line width=0.2mm
    },
  /s/descender/.style=
    {
      scolord,
      line cap=butt,
      line width=0.2mm,
      dash pattern=on 2*\nibwidth off \nibwidth,
      dash phase=(\depthy+(\ascenderNW+\descenderNW)*\nibwidth)/tan(\slantdegree)
    },
  pics/ystamp/.style=
    {
      code=
        {
          % Define basis
          \coordinate (u) at (\totalwidth,0);
          \coordinate (v) at (0,\nibwidth);

          % Draw guide lines (remove cap line if desired)
          \draw [/s/ascender]  (${\descenderNW+\ascenderNW}*(v)$) -- +(u);
          \draw [/s/cap]       (${\descenderNW+\capNW     }*(v)$) -- +(u);
          \draw [/s/xheight]   (${\descenderNW+\xheightNW }*(v)$) -- +(u);
          \draw [/s/baseline]  (${\descenderNW            }*(v)$) -- +(u);
          \draw [/s/descender] (${0                       }*(v)$) -- +(u);

          % Uncomment the next line for a circle at baseline start
          % \draw [/s/basecircle](${\descenderNW            }*(v)$) circle (\nibwidth/2);
        }
    }
}

% Declare a tikz pic for stamping left to right
\tikzset{
  /s/slant/guide/.style=
    {
      scolors,
      line cap=round,
      line width=0.2mm,
      dash pattern=on 0*\nibwidth off \nibwidth/sin(\slantdegree)
    },
  /s/slant/ends/.style=
    {
      scolors,
      fill=white,
      line cap=round,
      line width=0.2mm
    },
  pics/xstamp/.style=
    {
      code=
        {
          % Define begin and end points
          \coordinate (b) at (\sxoff,-\syoff);
          \coordinate (e) at ($(b)+(\slantdegree-180:{\slantlen})$);

          % Uncomment the next three lines to draw end arcs
          % \draw [/s/slant/ends]
          % ([xshift=-\nibwidth/2] b) arc ( 180:0:\nibwidth/2)
          % ([xshift=-\nibwidth/2] e) arc (-180:0:\nibwidth/2);

          % Draw slant guidelines
          \draw [/s/slant/guide] (b) -- (e);
        }
    }
}

%%%%%%%%%%%%%%%%%%%%%%%%%%%%%%%%%%%%%%%%%%%%%%%%%%%%%%%%%%%%%%%%%%%%%%%%%%%%%%%%

% Everything below this line is not intended to be user-configurable

% Macro for maximum length of slant guidelines; depends on \totalheight
\def\slantlen{(\totalheight-\syoff)/sin(\slantdegree)}

% Compute stamping lengths and separations
\pgfmathsetlengthmacro{\sxsep}{\xsepNW*\nibwidth}
\pgfmathsetlengthmacro{\sxoff}{\xoffsetNW*\nibwidth}
\pgfmathsetlengthmacro{\sylen}{(\ascenderNW+\descenderNW)*\nibwidth}
\pgfmathsetlengthmacro{\sysep}{\ysepNW*\nibwidth}
\pgfmathsetlengthmacro{\syoff}{\yoffsetNW*\nibwidth}

% Compute the number of stamps that can fit in printable area
\pgfmathsetmacro{\nx}{int((\textwidth+\textheight/tan(\slantdegree))/\sxsep)}
\pgfmathsetmacro{\ny}{int((\textheight+\sysep-\syoff)/(\sylen+\sysep))}

% Set PDF properties
\extractcolorspec{scolora}{\scolora}
\extractcolorspec{scolorc}{\scolorc}
\extractcolorspec{scolorx}{\scolorx}
\extractcolorspec{scolorb}{\scolorb}
\extractcolorspec{scolord}{\scolord}
\extractcolorspec{scolors}{\scolors}
\def\pdftitle{%
  \ny{} \template{} for \nibwidth{} nibs, with
  ascender  at \ascenderNW{}   nibs,
  caps      at \capNW{}        nibs,
  x-height  at \xheightNW{}    nibs,
  descender at -\descenderNW{} nibs,
  and \slantdegree\textdegree-guides separated by \xsepNW{} nibs
}
\hypersetup{
  pdftitle=\pdftitle,
  pdfkeywords=
    {%
      \template,
      a=\scolora; c=\scolorc,
      x=\scolorx; b=\scolorb,
      d=\scolord; s=\scolors,
    },
  pdfsubject={https://github.com/maverickwoo/paperpad-templates},
  pdfauthor={Maverick Woo}
}

% !TeX root = ./explain.tex
%
\ifshowbb\begin{tikzpicture}[remember picture, overlay, draw=black, very thick,
    dashed]

  % Draw rectangles around text areas
  \draw     (current page text area.south west)
  rectangle (current page text area.north east);
  \draw     (current page marginpar area.south west)
  rectangle (current page marginpar area.north east);
  \draw     (current page header area.south west)
  rectangle (current page header area.north east);
  \draw     (current page footer area.south west)
  rectangle (current page footer area.north east);

\end{tikzpicture}\fi
%
\begin{tikzpicture}[remember picture, overlay]

  % Add watermark into background
  \def\repourl{https://github.com/maverickwoo/paperpad-templates}
  \draw [inner sep=0pt, font=\ttfamily\tiny, color=wmcolor]
  node [anchor=north] at ($(current page.north) - (0mm,5mm)$) {
    \href{\repourl}{\number\year-\number\month-\number\day}}
  node [anchor=south] at ($(current page.south) + (0mm,5mm)$) {
    \href{\repourl}{\pdftitle}};

  % Create nodes "canvas rect" and "picture rect" using "current page text area"
  \begin{scope}[
      shift=(current page text area.north west),
      anchor=north west,
      line width=0, % node anchor points are on the outside of border lines
      fill opacity=\ifshowbb 0.5 \else 0 \fi]

    % Compute picture top-left (either center picture or anchor to canvas top-left)
    \pgfmathsetlengthmacro{\picturetop}{\canvasvcentering ?
      (\canvasheight-\pictureheight)/2 : \canvastop}
    \pgfmathsetlengthmacro{\pictureleft}{\canvashcentering ?
      (\canvaswidth-\picturewidth)/2 : \canvasleft}

    \node (canvas rect) [fill=cyan]
    at (\canvasleft, -\canvastop)
    [minimum width=\canvaswidth, minimum height=\canvasheight] {ct: \canvastop, pt: \picturetop};

    \node (picture rect) [fill=pink]
    at (\pictureleft, -\picturetop)
    [minimum width=\picturewidth, minimum height=\pictureheight] {ct: \canvastop, pt: \picturetop};

  \end{scope}

  % Draw target picture
  % (depend on picture rect, ny, nx, sylen, sysep, y-stamp, ...)
  \begin{scope}[shift=(picture rect.south west)]

    % Print y-stamps from top to bottom along picture west
    \foreach \iy in {\synum,...,1}
      {
        \pgfmathsetlengthmacro{\xcoord}{0}
        \pgfmathsetlengthmacro{\ycoord}{(\iy   -  1)*(\sylen+\sysep)}
        \pgfmathsetlengthmacro{\ydepth}{(\synum-\iy)*(\sylen+\sysep)}
        \draw (\xcoord,\ycoord) pic {y-stamp};
      }

    % Start clipping on picture east and west for x-stamps
    \clip     (\picturewidth,0 |- current page.north)
    rectangle (            0,0 |- current page.south);

    % Print x-stamps from left to right along picture north
    \foreach \ix in {1,...,\sxnum}
      {
        \pgfmathsetlengthmacro{\xcoord}{\sxoff+(\ix-1)*\sxsep}
        \pgfmathsetlengthmacro{\ycoord}{\pictureheight}
        \draw (\xcoord,\ycoord) pic {x-stamp};
      }
  \end{scope}

\end{tikzpicture}\ignorespaces

